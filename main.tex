\documentclass[conference]{IEEEtran}
\usepackage{graphicx}
\usepackage{cite}
\usepackage[utf8]{inputenc}

\begin{document}	
\title{Motores Elétricos}
\author{\IEEEauthorblockN{Marcelo Andrade\IEEEauthorrefmark{1} e Matheus Marotzke\IEEEauthorrefmark{2}}
\IEEEauthorblockA{Insper - Instituto de Ensino e Pesquisa} marceloga1@al.insper.edu.br\IEEEauthorrefmark{1}, matheusdmd@al.insper.edu.br\IEEEauthorrefmark{2}}

\maketitle

\begin{abstract}
Esse artigo visa mapear os diversos tipos de motores elétricos existentes, seus funcionamentos e peculiaridades. (...)
\end{abstract}


\section{Introdução}

\section{Motor universal}
\section{Motor de repulsão}

\section{Motor DC}
\subsection{Série}
\subsection{Paralelo}
\subsection{Composto}
\subsection{Derivativo}
\section{Motor DC de imãs permanentes}
\section{Motor DC sem escova}
\section{Motor de passo}
\section{Motor de indução}
\section{Motor AC assíncrono}


\section{}

\end{document}